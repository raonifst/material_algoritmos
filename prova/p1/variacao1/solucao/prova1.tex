\documentclass[11pt]{article}
\usepackage[brazil]{babel}
\usepackage{isolatin1}
\usepackage{charter}
\usepackage[usenames]{color}

\newcommand{\esp}{$\;\;\;\;\;$}
\setlength{\parskip}{.5pc} 
\setlength{\paperwidth}{216mm}  % tamanho carta
\setlength{\paperheight}{219mm}
\setlength{\textwidth}{41pc}
\setlength{\textheight}{60pc}
\setlength{\oddsidemargin}{-1cm}
\setlength{\evensidemargin}{-1cm}
\setlength{\topmargin}{-1cm}
\begin{document}
\parindent=0pt

\begin{center}
{\bf {\large Algoritmos e Programa��o de Computadores}} \\

1\raisebox{1ex}{\underline{a}} Prova
\end{center}\vspace*{-7mm}

\begin{minipage}{13cm}
\framebox[13cm][l]{{\bf Nome:}}

\vspace{\baselineskip}
\framebox[8cm][l]{{\bf RGA:}} 

\noindent
{\bf Instru��es:} {\bf N�o �
permitida consulta} a qualquer material. {\em Somente ser�o
consideradas respostas nos espa�os marcados.} Use os versos das folhas
como rascunho.
\end{minipage} \hfill 
\begin{minipage}{5cm}
\begin{center}
\begin{tabular}{|c|c|c|}
\hline
Quest�o & Valor & Nota \\ \hline \hline
1 & 2,0 & \\ \hline
2 & 2,0 & \\ \hline
3 & 2,0 & \\ \hline
4 & 2,0 & \\ \hline
5 & 2,0 & \\ \hline \hline
Total & 10,0 & \\ \hline
\end{tabular}\hspace*{-2cm}
\end{center}
\end{minipage}

\vspace*{-4mm}


\vspace{\baselineskip} {\bf 1.} Considere o c�digo a seguir e responda as perguntas abaixo.
\begin{verbatim}
#include <stdio.h>
int main()
{
   int i, j, n, m;
   scanf("%d %d", &n, &m);
   for(i = 1; i <= n; i++)	{
      for(j = i; j <= m; j++)	{
         printf("*");
      }
      printf("\n");
   }
   return 0;
}

\end{verbatim}

\vspace{\baselineskip} {\bf a)} Mostre a sa�da da execu��o deste programa para $m$ igual a 5 e $n$ igual a 5.  

{\Large\begin{tabular}{|l|}
\hline
{\textcolor{red}{\bf *****}} \\ \hline % 1
{\textcolor{red}{\bf ****}}\mbox{}\\ \hline %2
{\textcolor{red}{\bf ***}}\mbox{}\\ \hline %3
{\textcolor{red}{\bf **}} \mbox{}\\ \hline %4
{\textcolor{red}{\bf *}}\mbox{}\\ \hline %5
\mbox{}\hspace*{\textwidth}\\ \hline %6
\mbox{}\\ \hline %7
\end{tabular}}

\vspace{\baselineskip} {\bf b)} Supondo que os valores lidos da entrada s�o 7 e 7, respectivamente, quantas vezes este programa imprime o caractere asterisco (`*')?

{\Large\begin{tabular}{|l|}
\hline
\textcolor{red}{O caractere asterisco ser� impresso 28 vezes.} \\ \hline % 1
\hspace*{\textwidth}\mbox{}\\ \hline %2
\end{tabular}}\vspace*{-2cm} \newpage


\vspace{\baselineskip} {\bf 2.} Escreva um programa em linguagem C que calcula a soma dos 100 primeiros termos da seguinte s�rie:
\[s = \frac{1}{2} - \frac{3}{4} + \frac{5}{6} - \frac{7}{8} + ... \]
{\Large\begin{tabular}{|l|} \hline
\textcolor{red}{$\#$include $<$stdio.h$>$} \\ \hline % 1
\mbox{}\\ \hline %2
{\hspace{0cm}\textcolor{red}{int main( )}\mbox{}\\ \hline %3
{\hspace{0cm}\textcolor{red}{\{}\mbox{}\\ \hline %4
{\hspace{1.5cm}\textcolor{red}{double soma= 0.0;}\mbox{}\\ \hline %5
{\hspace{1.5cm}\textcolor{red}{int i, denominador = 2, numerador = 1, sinal = 1;}\mbox{}\\ \hline %6
\mbox{}\\ \hline %7
{\hspace{1.5cm}\textcolor{red}{for( i = 1; i $<= 100$ ; i++ )	\{}\mbox{}\\ \hline %8
{\hspace{3cm}\textcolor{red}{soma +=  sinal*(numerador+0.0)/denominador;}\mbox{}\\ \hline %9
{\hspace{3cm}\textcolor{red}{denominador += 2;}\mbox{}\\ \hline %10
{\hspace{3cm}\textcolor{red}{numerador += 2;}\mbox{}\\ \hline %11
{\hspace{3cm}\textcolor{red}{sinal *=-1;}\mbox{}\\ \hline %12
{\hspace{1.5cm}\textcolor{red}{\}}\mbox{}\\ \hline %13
\mbox{}\\ \hline %14
{\hspace{1.5cm}\textcolor{red}{printf(``\%lf\n'', soma); }\mbox{}\\ \hline %15
\mbox{}\\ \hline %16
{\hspace{1.5cm}\textcolor{red}{return 0;}\mbox{}\\ \hline %17
{\hspace{0cm}\textcolor{red}{\}}\mbox{}\\ \hline %18
\mbox{}\hspace*{\textwidth}\\ \hline %19
\mbox{}\\ \hline %15
\mbox{}\\ \hline %16
\mbox{}\\ \hline %17
\mbox{}\\ \hline %18
\mbox{}\\ \hline %19
\mbox{}\\ \hline %15
\mbox{}\\ \hline %16
\mbox{}\\ \hline %17
\mbox{}\\ \hline %17
\mbox{}\\ \hline %18
\mbox{}\\ \hline %19
\mbox{}\\ \hline 
\mbox{}\\ \hline 
\mbox{}\\ \hline 
\mbox{}\\ \hline 
\end{tabular}}\vspace*{-2cm}\newpage

\vspace{\baselineskip} {\bf 3.} Um n�mero inteiro � dito {\it abundante} se a soma dos seus divisores (excluindo o pr�prio n�mero)
� maior do que ele pr�prio. Por exemplo, o n�mero $12$ � denominado abundante pois a soma dos seus divisores (1, 2, 3, 4 e 6)
� igual a 16. Sabendo disso, escreva um programa em linguagem C que dado um n�mero, lido do teclado, imprima ``SIM" ~se este 
n�mero � abundante e ``NAO"~ caso contr�rio.
{\Large\begin{tabular}{|l|} \hline
\textcolor{red}{$\#$include $<$stdio.h$>$} \\ \hline % 1
\mbox{}\\ \hline %2
{\hspace{0cm}\textcolor{red}{int main( )}\mbox{}\\ \hline %3
{\hspace{0cm}\textcolor{red}{\{}\mbox{}\\ \hline %4
{\hspace{1.5cm}\textcolor{red}{int i, n, soma = 1;}\mbox{}\\ \hline %5
\mbox{}\\ \hline %6
{\hspace{1.5cm}\textcolor{red}{scanf(``\%d'',\&n);}\mbox{}\\ \hline %7
\mbox{}\\ \hline %8
{\hspace{1.5cm}\textcolor{red}{for(i=2; i $<=$ n/2 ; i++)~\{}\mbox{}\\ \hline %9
{\hspace{3cm}\textcolor{red}{if (n\%i == 0)}\mbox{}\\ \hline %10
{\hspace{4.5cm}\textcolor{red}{soma += i;}\mbox{}\\ \hline %11
{\hspace{1.5cm}\textcolor{red}{\}}\mbox{}\\ \hline %13
\mbox{}\\ \hline %13
{\hspace{1.5cm}\textcolor{red}{if(soma $>$ n)}\mbox{}\\ \hline %12
{\hspace{3cm}\textcolor{red}{printf("SIM\n");}\mbox{}\\ \hline %13
{\hspace{1.5cm}\textcolor{red}{else}\mbox{}\\ \hline %14
{\hspace{3cm}\textcolor{red}{printf("NAO\n");}\mbox{}\\ \hline %15
\mbox{}\\ \hline %16
{\hspace{1.5cm}\textcolor{red}{return 0;}\mbox{}\\ \hline %17
{\hspace{-0cm}\textcolor{red}{\}}\mbox{}\\ \hline %18
\mbox{}\hspace*{\textwidth}\\ \hline %19
\mbox{}\\ \hline %16
\mbox{}\\ \hline %17
\mbox{}\\ \hline %18
\mbox{}\\ \hline %19
\mbox{}\\ \hline %15
\mbox{}\\ \hline %18
\mbox{}\\ \hline %19
\mbox{}\\ \hline 
\mbox{}\\ \hline
\mbox{}\\ \hline 
\mbox{}\\ \hline 
\mbox{}\\ \hline 
\end{tabular}}\vspace*{-2cm}\newpage



\vspace{\baselineskip} {\bf 4.} Considere o seguinte c�digo que encontra e imprime o menor n�mero 
em uma sequ�ncia de $n$ n�meros digitados.
\begin{verbatim}
1  #include <stdio.h>
2  int main()	{
3     int numero, i, n, menor;	
4     printf("Informe a quantidade de numeros: ");
5     scanf("%d", &n);	
6     if(n<=0){
7        printf("Quantidade invalida\n");
8        return 1;
9     }	
10    printf("Informe um numero: ");
11    scanf("%d", &numero);
12    menor = numero;
13    for( i = 2; i <=n; i++)	{
14       printf("Informe um numero: ");
15       scanf("%d", &numero);
16       if(numero < menor)
17          menor = numero;
18    }
19    printf("O menor numero digitado foi: %d\n", menor);	
20    return 0;
21  }

\end{verbatim}

\vspace{\baselineskip} {\bf a)} Mostre uma sequ�ncia de $5$ n�meros para qual, em uma execu��o deste programa, a quantidade de vezes 
em que a instru��o da linha $17$ � executada � m�xima.

{\Large\begin{tabular}{|l|} \hline
 \textcolor{red}{Qualquer sequ�ncia estritamente decrescente como, por exemplo, 5, 4, 3, 2, 1.}\\ \hline % 
\end{tabular}}

\vspace{\baselineskip} {\bf b)} Mostre o que deve ser alterado neste c�digo para que o maior n�mero da sequ�ncia seja
encontrado e impresso.

{\Large\begin{tabular}{|l|} \hline
\textcolor{red}{03 - int numero, i, n, maior;} \\ \hline
\textcolor{red}{12 - maior = numero;} \\ \hline
\textcolor{red}{16 - if(numero $>$ maior)} \\ \hline
\textcolor{red}{17 - maior = numero;} \\ \hline
\textcolor{red}{19 - printf(``O maior numero digitado foi: \%d$\backslash${\tt n}'', maior);} \\ \hline
\hspace*{\textwidth} \\ \hline % 6
\hspace*{\textwidth} \\ \hline % 7
\end{tabular}}\newpage


\vspace{\baselineskip} {\bf 5.} Escreva um programa em linguagem C que dado um n�mero inteiro $n$, lido do teclado,
calcula e imprime a soma dos algorismos deste n�mero. 
Por exemplo, para $n$ valendo $867$ deve-se imprimir $21$ que � o resultado da soma de $8$, $6$ e $7$. \\
{\Large\begin{tabular}{|l|} \hline
\textcolor{red}{$\#$include $<$stdio.h$>$} \\ \hline % 1
\mbox{}\\ \hline %2
\textcolor{red}{int main( )}\mbox{}\\ \hline %3
\textcolor{red}{\{}\mbox{}\\ \hline %4
\hspace{1.5cm}\textcolor{red}{int n, soma = 0;}\mbox{}\\ \hline %5
\hspace{1.5cm}\textcolor{red}{scanf(``\%d'',\&n);}\mbox{}\\ \hline %6
\mbox{}\\ \hline %7
\hspace{1.5cm}\textcolor{red}{while(n $>$ 0)~\{}\mbox{}\\ \hline %8
\hspace{3cm}\textcolor{red}{soma += n \% 10;} \mbox{}\\ \hline %9
\hspace{3cm}\textcolor{red}{n = n / 10;}\mbox{}\\ \hline %10
\hspace{1.5cm}\textcolor{red}{\}}\mbox{}\\ \hline %11
\mbox{}\\ \hline %12
\hspace{1.5cm}\textcolor{red}{printf(``\%d\n'', soma);}\mbox{}\\ \hline %13
\hspace{1.5cm}\textcolor{red}{return 0;}\mbox{}\\ \hline %14
\textcolor{red}{\}}\mbox{}\\ \hline %15
\mbox{}\\ \hline %16
\mbox{}\\ \hline %17
\mbox{}\\ \hline %18
\mbox{}\\ \hline %19
\mbox{}\\ \hline %15
\mbox{}\\ \hline %16
\mbox{}\\ \hline %17
\mbox{}\\ \hline %18
\mbox{}\\ \hline %19
\mbox{}\\ \hline %15
\hspace*{\textwidth}\mbox{}\\ \hline %16
\mbox{}\\ \hline %17
\mbox{}\\ \hline %17
\mbox{}\\ \hline %18
\mbox{}\\ \hline %19
\mbox{}\\ \hline 
\mbox{}\\ \hline
\mbox{}\\ \hline 
\mbox{}\\ \hline 
\end{tabular}}\vspace*{-2cm}\newpage





\end{document}
