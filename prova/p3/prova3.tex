\documentclass[11pt]{article}
\usepackage[brazil]{babel}
\usepackage{isolatin1}
\usepackage{charter}
\usepackage{graphicx}
\usepackage{pstricks}
\usepackage{fancybox}
\usepackage{tikz}


\usepackage{color, colortbl}
\definecolor{Gray}{gray}{0.7}


\newcommand{\esp}{$\;\;\;\;\;$}
\setlength{\parskip}{.5pc} 
\setlength{\paperwidth}{216mm}  % tamanho carta
\setlength{\paperheight}{219mm}
\setlength{\textwidth}{41pc}
\setlength{\textheight}{60pc}
\setlength{\oddsidemargin}{-1cm}
\setlength{\evensidemargin}{-1cm}
\setlength{\topmargin}{-1cm}
\begin{document}
\parindent=0pt

\begin{center}
{\bf {\large Algoritmos e Programa��o de Computadores}} \\

3\raisebox{1ex}{\underline{a}} Prova
\end{center}\vspace*{-7mm}

\begin{minipage}{13cm}
\framebox[13cm][l]{{\bf Nome:}}
\vspace{\baselineskip}
\framebox[8cm][l]{{\bf RGA:}} 

\noindent
{\bf Instru��es:} {\bf N�o �
permitida consulta} a qualquer material. {\em Somente ser�o
consideradas respostas nos espa�os marcados.} Use os versos das folhas
como rascunho.
\end{minipage} \hfill 
\begin{minipage}{5cm}
\begin{center}
\begin{tabular}{|c|c|c|}
\hline
Quest�o & Valor & Nota \\ \hline \hline
1 & 3,0 & \\ \hline
2 & 2,5 & \\ \hline
3 & 2,5 & \\ \hline
4 & 2,0 & \\ \hline \hline
Total & 10,0 & \\ \hline
\end{tabular}\hspace*{-2cm}
\end{center}
\end{minipage}

\vspace*{-4mm}


\vspace{\baselineskip} {\bf 1.} O programa a seguir faz a leitura de seu RGA e 
armazena os quatro �ltimos d�gitos em um vetor de inteiros. Este vetor � utilizado para preencher uma matriz de inteiros.
\begin{verbatim}
1 #include <stdio.h>
2
3 int main() {
4     int vetor[4], matriz[4][4], i, j, coluna, x, maior;
5     long rga;
6     scanf("%ld", &rga);
7     for(i = 4; i > 0; i--)  { /* Leitura dos quatros �ltimos d�gitos do seu RGA */
8          vetor[i-1] = rga % 10;
9          rga /= 10;
10    }
11
12    /* In�cio da etapa de preenchimento */
13    for(i = 0;  i < 4; i++) 
14         for(j = 0; j < 4; j++)
15              matriz[i][j] = i + j; 
16
17    for(i = 0;  i < 4; i++) 
18         matriz[i][i] = vetor[i]; 
19    /* Fim da etapa de preenchimento */
20                              
21    coluna = 3;
22    maior = matriz[3][coluna]; 
23    for(i = 3; i >= 0; i--) {
24         for(j = 3; j >= 0; j--) {
25              if(matriz[i][j] > maior) {
26                   maior = matriz[i][j];
27                   coluna = j;
28              }
29         }
30     }
31     x = 0;
32     for(i = 0; i < 4; i++)
33          x += matriz[i][coluna];
34
35     printf("x = %d\n", x);
36     return 0;
37 }
\end{verbatim}
\newpage
 {\bf a)} Mostre a matriz obtida por este programa ap�s a etapa de preenchimento.  

{\Large\begin{tabular}{|c|}
\hline
\hspace*{\textwidth} \\ \hline % 1
\mbox{}\\ \hline %2
\mbox{}\\ \hline %3
\mbox{}\\ \hline %4
\mbox{}\\ \hline %5
\mbox{}\\ \hline %6
\end{tabular}}\\

{\bf b)} Qual o valor � impresso na linha 35? O que a vari�vel {\it x} representa?  

{\Large\begin{tabular}{|c|}
\hline
\hspace*{\textwidth} \\ \hline % 1
\mbox{}\\ \hline %2
\mbox{}\\ \hline %3
\mbox{}\\ \hline %4
\mbox{}\\ \hline %5
\mbox{}\\ \hline %6
\end{tabular}}\\

 {\bf c)} Mostre uma matriz de tamanho $4\times 4$, para a qual, em uma execu\c{c}�o deste c�digo, � m�xima a quantidade de vezes em que as instru\c{c}�es das linhas 26 e 27 s�o executadas.
 
{\Large\begin{tabular}{|c|}
\hline
\hspace*{\textwidth} \\ \hline % 1
\mbox{}\\ \hline %2
\mbox{}\\ \hline %3
\mbox{}\\ \hline %4
\mbox{}\\ \hline %5
\mbox{}\\ \hline %6
\end{tabular}}

\newpage


\vspace{\baselineskip} {\bf 2.} A seguinte fun\c{c}�o recursiva recebe um n�mero inteiro positivo $n$ e um vetor $v$ contendo $n$ n�meros inteiros em ordem crescente. Preencha os espa\c{c}os deixados em branco para que a fun\c{c}�o verifique se $v$ possui n�meros repetidos.  Os valores $1$ e $0$ devem ser devolvidos, respectivamente, para indicar se h� ou n�o repeti\c{c}�es.\\


\verb+1 int rep(int v[], int n)  {+\\
\verb+     +\\
\verb+2     if(+ \ovalbox{\begin{minipage}{5cm}\hfill\vspace{0.8cm}\end{minipage}} \verb+) //Base da recurs�o+ \\ 
\verb+3          return 0;+\\
\verb+    +\\
\verb+4     return+ \ovalbox{\begin{minipage}{5cm}\hfill\vspace{0.8cm}\end{minipage}} \verb+||+ \ovalbox{\begin{minipage}{5cm}\hfill\vspace{0.8cm}\end{minipage}}\verb+;+\\
\verb+5 }+\\
%%4     int vetor[4], matriz[4][4], i, j, coluna, x, maior;
%\end{verbatim}

%Observe que os espa\c{c}os


\newpage

\vspace{\baselineskip} {\bf 3.} Escreva uma fun\c{c}�o recursiva que recebe dois n�meros inteiros $a$ e $b$ como par�metro e calcula, utilizando somente subtra\c{c}�es, o resultado da divis�o inteira de $a$ por $b$. Para $a = 10$ e $b = 3$, por exemplo, a fun\c{c}�o deve devolver $3$.

{\Large\begin{tabular}{|l|} \hline
\\ \hline % 1
\hspace*{\textwidth} \mbox{}\\ \hline %2
\mbox{}\\ \hline %3
\mbox{}\\ \hline %4
\mbox{}\\ \hline %5
\mbox{}\\ \hline %6
\mbox{}\\ \hline %7
\mbox{}\\ \hline %8
\mbox{}\\ \hline %9
\mbox{}\\ \hline %10
\mbox{}\\ \hline %11
\mbox{}\\ \hline %12
\mbox{}\\ \hline %13
\mbox{}\\ \hline %14
\mbox{}\\ \hline %15
\mbox{}\\ \hline %16
\mbox{}\\ \hline %17
\mbox{}\\ \hline %18
\mbox{}\\ \hline %19
\mbox{}\\ \hline %15
\mbox{}\\ \hline %16
\mbox{}\\ \hline %17
\mbox{}\\ \hline %18
\mbox{}\\ \hline %19
\mbox{}\\ \hline %15
\mbox{}\\ \hline %16
\mbox{}\\ \hline %17
\mbox{}\\ \hline %17
\mbox{}\\ \hline %18
\mbox{}\\ \hline %19
\mbox{}\\ \hline 
\mbox{}\\ \hline
\mbox{}\\ \hline 
\end{tabular}}\vspace*{-2cm}\newpage


\vspace{\baselineskip} {\bf 4.} O programa a seguir recebe uma cadeia de caracteres contendo uma mensagem escondida. A mensagem foi codificada considerando as diagonais de uma matriz de tamanho $n \times n$ (veja exemplo). O programa dever� mostrar a mensagem na tela. As fun\c{c}�es {\tt main} e {\tt tamanho\_frase} s�o dadas. Implemente as outras duas fun\c{c}�es.\\

\hline
\noindent Exemplo de mensagem de entrada\\
{\tt 1lcpskl+y5mcpj3bpbsllcrehfmeokecwitsnglrex(tj(xypynbqfxx)uvgfaj)}\\
\hline

\noindent Mensagem em formato matricial

\begin{center}
\begin{tabular}{|c|c|c|c|c|c|c|c|}
\hline
\cellcolor{Gray} 1 & l & c & p & s & k & l & \cellcolor{Gray}+ \\ \hline
y & \cellcolor{Gray} 5 & m & c & p & j & \cellcolor{Gray}3 & b \\ \hline
p & b & \cellcolor{Gray} s & l & l & \cellcolor{Gray}c & r & e \\ \hline
h & f & m & \cellcolor{Gray} e & \cellcolor{Gray}o & k & e & c \\ \hline
w & i & t & \cellcolor{Gray}s & \cellcolor{Gray} n & g & l & r \\ \hline
e & x & \cellcolor{Gray}( & t & j & \cellcolor{Gray}( & x & y \\ \hline
p & \cellcolor{Gray}y & n & b & q & f & \cellcolor{Gray}x & x \\ \hline
\cellcolor{Gray}) & u & v & g & f & a & j & \cellcolor{Gray}) \\ \hline
\end{tabular}
\end{center}

\hline
\noindent Mensagem escondida

{\tt 15sen(x)+3cos(y)}\\
\hline


\begin{verbatim}
#include <stdio.h>
#include <math.h>

int tamanho_frase(char frase[]);
void preencher_matriz(char frase[], char matriz[100][100], int n);
void mostrar_mensagem(char matriz[100][100], int n);

int main()
{
    char matriz[100][100];
    char frase[10000];
    int n;

    scanf("%s", frase);
    n = sqrt(tamanho_frase(frase));
    preencher_matriz(frase, matriz, n);
    mostrar_mensagem(matriz, n);

    return 0;
}

\end{verbatim}
\newpage
\begin{verbatim}
int tamanho_frase(char frase[])
{
    int i = 0;
    while(frase[i]!= `\0') i++;
    return i;
}

\end{verbatim}
 {\bf a)} Implemente a fun\c{c}�o {\tt preencher\_matriz}, que recebe uma cadeia de caracteres de tamanho $n \times n$ e preenche a matriz com estes caracteres.
 
{\Large\begin{tabular}{|l|} \hline
{\Small\verb+void preencher_matriz(char frase[], char matriz[100][100], int n) { +} \\ \hline % 1
\hspace*{\textwidth} \mbox{}\\ \hline %2
\mbox{}\\ \hline %3
\mbox{}\\ \hline %4
\mbox{}\\ \hline %5
\mbox{}\\ \hline %6
\mbox{}\\ \hline %7
\mbox{}\\ \hline %8
\mbox{}\\ \hline %9
\mbox{}\\ \hline %10
\mbox{}\\ \hline %11
\verb+}+\\ \hline
\end{tabular}}
\\ 

{\bf b)} Implemente a fun\c{c}�o {\tt mostrar\_mensagem} que imprime a mensagem escondida nas diagonais da matriz.

{\Large\begin{tabular}{|l|} \hline
{\Small\verb+void mostrar_mensagem(char matriz[100][100], int n) {+} \\ \hline % 1
\hspace*{\textwidth} \mbox{}\\ \hline %2
\mbox{}\\ \hline %3
\mbox{}\\ \hline %4
\mbox{}\\ \hline %5
\mbox{}\\ \hline %6
\mbox{}\\ \hline %7
\mbox{}\\ \hline %8
\mbox{}\\ \hline %9
\mbox{}\\ \hline %10
\mbox{}\\ \hline %11
\mbox{}\\ \hline %12
\mbox{}\\ \hline %12
\verb+}+\\ \hline
\end{tabular}}



\end{document}
