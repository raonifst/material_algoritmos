\documentclass[11pt]{article}
\usepackage[brazil]{babel}
\usepackage{isolatin1}
\usepackage{charter}
\usepackage[all]{xy}
\usepackage{graphicx}


\newcommand{\esp}{$\;\;\;\;\;$}
\setlength{\parskip}{.5pc} 
\setlength{\paperwidth}{190mm}  % tamanho carta
\setlength{\paperheight}{219mm}
\setlength{\textwidth}{41pc}
\setlength{\textheight}{60pc}
\setlength{\oddsidemargin}{-1cm}
\setlength{\evensidemargin}{-1cm}
\setlength{\topmargin}{-1cm}
\begin{document}
\parindent=0pt

\begin{center}
{\bf {\large Universidade Federal de Mato Grosso}} \\
{\bf {\large Instituto de Engenharia}} \\ 
{\bf {\large Algoritmos e Programa��o de Computadores}} \\


Segundo Semestre de 2014

Casos de Testes do 1\raisebox{1ex}{\underline{{\tiny o}}} Trabalho de Pr�tico
\end{center}\vspace*{-5mm}





\vspace*{0.5cm}


Os dez casos de testes do primeiro trabalho pr�tico s�o apresentados nas Tabelas 1-10, a seguir.

\begin{table}[h!]
\begin{center}
   \begin{tabular}{|l|l| }
   \hline
 	Entrada	& Sa�da esperada | Tela do computador \\ \hline
    {\tt 40} & {\tt Nova Cacique: 17.} \\
   & \\ \hline
   \end{tabular} \\

\end{center}
\caption{Primeiro caso de teste.}
\end{table}

\begin{table}[h!]
\begin{center}
   \begin{tabular}{|l|l| }
   \hline
 	Entrada	& Sa�da esperada | Tela do computador \\ \hline
    {\tt 500} & {\tt Nova Cacique: 489.} \\
   & \\ \hline
   \end{tabular} \\

\end{center}
\caption{Segundo caso de teste.}
\end{table}

\begin{table}[h!]
\begin{center}
   \begin{tabular}{|l|l| }
   \hline
 	Entrada	& Sa�da esperada | Tela do computador \\ \hline
    {\tt 12} & {\tt Nova Cacique: 9.} \\
   & \\ \hline
   \end{tabular} \\

\end{center}
\caption{Terceiro caso de teste.}
\end{table}

\begin{table}[h!]
\begin{center}
   \begin{tabular}{|l|l| }
   \hline
 	Entrada	& Sa�da esperada | Tela do computador \\ \hline
    {\tt 499} & {\tt Nova Cacique: 487.} \\
   & \\ \hline
   \end{tabular} \\

\end{center}
\caption{Quarto caso de teste.}
\end{table}

\begin{table}[h!]
\begin{center}
   \begin{tabular}{|l|l| }
   \hline
 	Entrada	& Sa�da esperada | Tela do computador \\ \hline
    {\tt 45} & {\tt Nova Cacique: 27.} \\
   & \\ \hline
   \end{tabular} \\

\end{center}
\caption{Quinto caso de teste.}
\end{table}

\begin{table}[h!]
\begin{center}
   \begin{tabular}{|l|l| }
   \hline
 	Entrada	& Sa�da esperada | Tela do computador \\ \hline
    {\tt 98} & {\tt Nova Cacique: 69.} \\
   & \\ \hline
   \end{tabular} \\

\end{center}
\caption{Sexto caso de teste.}
\end{table}

\begin{table}[h!]
\begin{center}
   \begin{tabular}{|l|l| }
   \hline
 	Entrada	& Sa�da esperada | Tela do computador \\ \hline
    {\tt 2} & {\tt Nova Cacique: 1.} \\
   & \\ \hline
   \end{tabular} \\

\end{center}
\caption{S�timo caso de teste.}
\end{table}

\begin{table}[h!]
\begin{center}
   \begin{tabular}{|l|l| }
   \hline
 	Entrada	& Sa�da esperada | Tela do computador \\ \hline
    {\tt 9} & {\tt Nova Cacique: 3.} \\
   & \\ \hline
   \end{tabular} \\

\end{center}
\caption{Oitavo caso de teste.}
\end{table}

\begin{table}[h!]
\begin{center}
   \begin{tabular}{|l|l| }
   \hline
 	Entrada	& Sa�da esperada | Tela do computador \\ \hline
    {\tt 99} & {\tt Nova Cacique: 71.} \\
   & \\ \hline
   \end{tabular} \\

\end{center}
\caption{Nono caso de teste.}
\end{table}

\begin{table}[h!]
\begin{center}
   \begin{tabular}{|l|l| }
   \hline
 	Entrada	& Sa�da esperada | Tela do computador \\ \hline
    {\tt 300} & {\tt Nova Cacique: 89.} \\
   & \\ \hline
   \end{tabular} \\

\end{center}
\caption{D�cimo caso de teste.}
\end{table}


\newpage






\end{document}
