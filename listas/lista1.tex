\documentclass[10pt]{article}

\usepackage[latin1]{inputenc}
\usepackage[portuges]{babel}
\usepackage{fullpage}
\usepackage{url}
\usepackage{ulem}

\def\exerc{\mathop{\it exerc}\nolimits}
\def\prova{\mathop{\it prova}\nolimits}
\def\prog{\mathop{\it prog}\nolimits}
\def\parcial{\mathop{\it parcial}\nolimits}
\def\final{\mathop{\it final}\nolimits}

\newcommand{\quest}[1] {\noindent {\bf {#1})}}
\newcommand{\prob}[3] {{\bf ENTRADA:} {\tt #1}

{\bf SA�DA:} {\tt #2}

{\bf DICA:} {#3\\}}

\newcommand{\proba}[3] {{\bf ENTRADA:} {\tt #1}

{\bf SA�DA:} {\tt #2}

{\bf DESCRI��O:} {#3\\}}


\pagestyle{empty}

\title{ {\footnotesize
    \hrule\vspace{1pt}\hrule\vspace{1ex}
    Instituto de Engenharia \hfill Universidade Federal de Mato Grosso
    \smallskip 
    \hrule\vspace{1pt}\hrule}\vspace{10pt}
  Algoritmos e Programa��o de Computadores \\[-6pt]}

\author{} 


\date{\bf Segundo Semestre de 2014}

\begin{document}
 
\maketitle


%\vspace{-0.5cm}

%{\Large \centering

%{\bf \normalsize Maiores Informa��es} 

%\url{http://www.students.ic.unicamp.br/~ra089067/mc102/index.html}

%}

%\vspace{0.5cm}
\thispagestyle{empty}


\begin{center} \Large \bf Exerc�cios \end{center}

\quest{1} O barquinho de um campon�s comporta apenas um item, al�m dele pr�prio. O barquinho pode levar e  trazer itens, respeitando as seguintes regras:

\begin{itemize}
\item O lobo devora a ovelha se os dois ficarem sozinhos e
\item A ovelha come o repolho se ficar sozinha com ele.
\end{itemize}

O objetivo do campon�s e atravessar o lobo, a ovelha e o repolho de uma margem de um rio para a outra. Encontre uma sequ�ncia de movimentos que pode ser utilizada para que ele atravesse em seguran�a.
\\

\quest{2} Suponha que voc� tenha dois jarros, um de cinco litros e outro de tr�s. Suponha tamb�m que voc� tenha uma fonte inesgot�vel de �gua. Encontre uma sequ�ncia de movimentos de encher e esvaziar jarros, de maneira que, ao final da sequ�ncia, voc� tenha quatro litros de �gua dentro do jarro de cinco litros.
\\

\quest{3} Suponha que voc� tenha 1000 moedas e 10 caixas pretas vazias.  Distribua as 1000 moedas nas caixas pretas e rotule cada uma delas com sua quantidade de moedas. Esta organiza��o deve permitir obter qualquer valor de 1 at� 1000, apenas combinando as caixas, sem alterar o seu conte�do.



%\bibliographystyle{abbrv}{\bibliography{mc}}

\end{document}

%%% Local Variables: 
%%% mode: latex
%%% portug-mode: t
%%% TeX-master: t
%%% End: 
