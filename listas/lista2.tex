\documentclass[10pt]{article}

\usepackage[latin1]{inputenc}
\usepackage[portuges]{babel}
\usepackage{fullpage}
\usepackage{url}
\usepackage{ulem}

\def\exerc{\mathop{\it exerc}\nolimits}
\def\prova{\mathop{\it prova}\nolimits}
\def\prog{\mathop{\it prog}\nolimits}
\def\parcial{\mathop{\it parcial}\nolimits}
\def\final{\mathop{\it final}\nolimits}

\newcommand{\quest}[1] {\noindent {\bf {#1})}}
\newcommand{\prob}[3] {{\bf ENTRADA:} {\tt #1}

{\bf SA�DA:} {\tt #2}

{\bf DICA:} {#3\\}}

\newcommand{\proba}[3] {{\bf ENTRADA:} {\tt #1}

{\bf SA�DA:} {\tt #2}

{\bf DESCRI��O:} {#3\\}}


\pagestyle{empty}

\title{ {\footnotesize
    \hrule\vspace{1pt}\hrule\vspace{1ex}
    Instituto de Engenharia \hfill Universidade Federal de Mato Grosso
    \smallskip 
    \hrule\vspace{1pt}\hrule}\vspace{10pt}
  Algoritmos e Programa��o de Computadores \\[-6pt]}



\date{\bf Segundo Semestre de 2014}

\begin{document}
 
\maketitle


%\vspace{-0.5cm}

%{\Large \centering

%{\bf \normalsize Maiores Informa��es} 

%\url{http://www.students.ic.unicamp.br/~ra089067/mc102/index.html}

%}

%\vspace{0.5cm}
\thispagestyle{empty}


\begin{center} \Large \bf Exerc�cios \end{center}

\quest{1} Escreva um programa que imprime ``Instituto de Engenharia"  na Tela.
\\

\quest{2} Escreva um programa que l� um n�mero inteiro {\it n} e imprime a tabuada deste n�mero na tela. Para n = 3, por exemplo, a seguinte sa�da deve ser produzida.\\
3 x 1 = 3\\
3 x 2 = 6\\
. . .\\
3 x 9 = 27\\
3 x 10 = 30\\
\\

\quest{3} Escreva um programa que l� um valor de velocidade em quil�metros por hora e imprime este valor em metros por segundo.
\\

\quest{4} Escreva um programa que l� a {\it base} e a {\it altura} de um tri�ngulo e imprime sua �rea.
\\

\quest{5} Escreva um programa que l� o {\it raio} de uma circunfer�ncia e imprime sua �rea.
\\

\quest{6} Escreva um programa que l� duas temperaturas, uma em Celsius e outra em Fahrenheit, e converte a primeira para Fahrenheit e a segunda para Celsius. Seu programa deve imprimir as quatro temperaturas, explicando as rela��es entre elas.
\\

\quest{7} Escreva um programa que l� 5 n�meros inteiros e imprime a m�dia aritm�tica destes 5 n�meros.
\\

\quest{8} Escreva uma outra vers�o do programa anterior utilizando apenas 2 vari�veis.
\\

\quest{9} Escreva um programa que l� as coordenadas de dois pontos no plano euclidiano e imprime a dist�ncia entre eles.
\\

\quest{10} Escreva um programa que  l� o n�mero de termos, o primeiro termo e raz�o de uma P.A (progress�o aritm�tica). Seu programa deve imprimir o valor do n-�simo termo e a soma dos n termos.
\\

\quest{11} Escreva uma vers�o do programa anterior para progress�o geom�trica.


%\bibliographystyle{abbrv}{\bibliography{mc}}

\end{document}

%%% Local Variables: 
%%% mode: latex
%%% portug-mode: t
%%% TeX-master: t
%%% End: 
