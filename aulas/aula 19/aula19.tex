\documentclass{beamer}

\usetheme{Boadilla}
%\usetheme{Warsaw}
%\setbeamercovered{transparent}
\beamertemplatetransparentcoveredhigh
\usepackage[portuges]{babel}
\usepackage[latin1]{inputenc}
\usepackage{ulem}

\newcommand{\eng}[1]{\textsl{#1}}
\newcommand{\cod}[1]{\texttt{#1}}

\title[9002 --- Aula 19]{9002 --- Aula 19\\
   Algoritmos e Programa��o de Computadores}

\author[IEng - UFMT]{Instituto de Engenharia -- UFMT}

\institute[2014/2]{Segundo Semestre de 2014}

\date{01 de dezembro de 2014}


\begin{document}

\begin{frame}[plain]
  \titlepage
\end{frame}

\begin{frame}
  \frametitle{Roteiro}
  \tableofcontents
\end{frame}



%%%%%%%%%%%%%%%%%%%%%%%%%%%%%%%%%%%%%%%%%%%%%%%%%%%
%%%%%%%%%%%%%%%%%%%%%%%%%%%%%%%%%%%%%%%%%%%%%%%%%%%
%%%%%%%%%%%%%%%%%%%%%%%%%%%%%%%%%%%%%%%%%%%%%%%%%%%
%%%%%%%%%%%%%%%%%%%%%%%%%%%%%%%%%%%%%%%%%%%%%%%%%%%
%%%%%%%%%%%%%%%%%%%%%%%%%%%%%%%%%%%%%%%%%%%%%%%%%%%

\section{Introdu\c{c}�o}

\begin{frame}[containsverbatim]
\frametitle{Cadeias de Caracteres} 
\begin{itemize}
\item A linguagem C n�o possui, explicitamente, o tipo para {\it cadeias de caracteres}.
\item Podemos, por�m, considerar um vetor de caracteres como uma {\it cadeia}.
\end{itemize}
\end{frame}

\begin{frame}[containsverbatim]
\frametitle{Cadeias de Caracteres} 
\begin{itemize}
\item Em C, uma cadeia de caracteres � sempre terminada pelo caracter especial:  '\textbackslash0'
\item {\bf Portanto, sempre, reserve um espa\c{c}o para um caracter a mais do que precisa!}
  \begin{itemize}
    \item Se por exemplo estivermos trabalhando com strings de 10 caracteres:
    \begin{verbatim}
     char st[11];
    \end{verbatim}
  \end{itemize} 

\end{itemize}
\end{frame}

%%%%%%%%%%%%%%%%%%%%%%%%%%%%%%%%%%%%%%%%%%%%%%%%%%%
\begin{frame}[containsverbatim]
\frametitle{Cadeias de Caracteres} 
\begin{itemize}
\item Para ler ou imprimir usamos o operador especial $\%s$.
\begin{verbatim}
#include <stdio.h>
int main(){
  char nome[80];

  printf("\nEntre com nome:");
  scanf("%s", nome);
  
  printf("O nome digitado foi: %s", nome);
  
  return 0;
}
\end{verbatim}
\item {\bf Note que neste caso n�o � utilizado o $\&$ no comando {\it scanf}}.
\end{itemize}
\end{frame}


%%%%%%%%%%%%%%%%%%%%%%%%%%%%%%%%%%%%%%%%%%%%%%%%%%%
\begin{frame}[containsverbatim]
\frametitle{Cadeias de Caracteres} 
\begin{itemize}
\item Para ler {\bf incluindo espa�os} use a op��o:    \%[ \^{}   \textbackslash n ].
\begin{verbatim}
#include <stdio.h>
int main(){
  char nome[80];

  printf("\nEntre com nome:");
  scanf("%[^\n]", nome);

  printf("O nome digitado foi: %s", nome);
  
  return 0;
}
\end{verbatim}
\end{itemize}
\end{frame}


\begin{frame}[containsverbatim]
\frametitle{Cadeias de Caracteres} 
\begin{itemize}
\item Pode-se utilizar ainda a fun\c{c}�o {\bf fgets}.
\begin{verbatim}
#include <stdio.h>
int main(){
  char nome[80];
  
  printf("\nEntre com nome:");
  fgets(nome, 80, stdin);

  printf("O nome digitado foi: %s", nome);
  
  return 0;
}
\end{verbatim}
\end{itemize}
\end{frame}


%%%%%%%%%%%%%%%%%%%%%%%%%%%%%%%%%%%%%%%%%%%%%%%%%%%
%%%%%%%%%%%%%%%%%%%%%%%%%%%%%%%%%%%%%%%%%%%%%%%%%%%
%%%%%%%%%%%%%%%%%%%%%%%%%%%%%%%%%%%%%%%%%%%%%%%%%%%
%%%%%%%%%%%%%%%%%%%%%%%%%%%%%%%%%%%%%%%%%%%%%%%%%%%
%%%%%%%%%%%%%%%%%%%%%%%%%%%%%%%%%%%%%%%%%%%%%%%%%%%

\section{Exemplos}


\begin{frame}[containsverbatim]
\frametitle{Exemplos} 
Como calcular o tamanho de uma cadeia de caracteres?
\end{frame}

\begin{frame}[containsverbatim]
\frametitle{Exemplos} 
\begin{scriptsize}
\begin{verbatim}
#include <stdio.h>

int tamanho(char str[])
{
  int tam = 0;
  while(str[tam] != '\0')
    tam++;
  
  return tam;    
}

int main()
{
  char palavra[100];

  scanf("%s", palavra);  
  printf("Voce digitou uma palavra com %d letras", tamanho(palavra));	
	
  return 0;
}

\end{verbatim}
\end{scriptsize}
\end{frame}



%%%%%%%%%%%%%%%%%%%%%%%%%%%%%%%%%%%%%%%%%%%%%%%%%%%
\begin{frame}[containsverbatim]
\frametitle{Exemplos} 
\begin{itemize}
\item Como ler uma cadeia de carateres e imprimir sua inversa? 
Assuma que a cadeia tem no m�ximo 80 caracteres.

\end{itemize}
\end{frame}


%%%%%%%%%%%%%%%%%%%%%%%%%%%%%%%%%%%%%%%%%%%%%%%%%%%
\begin{frame}[containsverbatim]
\frametitle{Exemplos} 
\begin{scriptsize}
\begin{verbatim}
#include <stdio.h>
int main(){
  char st[80], stInv[80];
  int tam, i, j;

  printf("Entre com o string: ");
  scanf("%s",st);

  tam = tamanho(st);
  
  stInv[tam] = '\0';

  j = tam-1;
  i = 0;
  while(i<tam){
    stInv[j] = st[i];
    i++;
    j--;
  }

  printf("A inversa e: %s\n",stInv);
  return 0;
}
\end{verbatim}
\end{scriptsize}
\end{frame}


%%%%%%%%%%%%%%%%%%%%%%%%%%%%%%%%%%%%%%%%%%%%%%%%%%%
\begin{frame}[containsverbatim]
\frametitle{Exemplos} 
Ou...
\begin{scriptsize}
\begin{verbatim}
#include <stdio.h>
int main(){
  char st1[80], stInversa[80];
  int i, j , tam;

  printf("Digite um texto (max. 80):");
  scanf("%s",st1);

  tam = tamanho(st);

  stInversa[tam] = '\0';
  for(j = tam-1, i = 0 ; j >= 0 ;  j--, i++){
    stInversa[j] = st1[i];
  }
  
  printf("A inversa e: %s\n", stInversa);
  return 0;
}
\end{verbatim}
\end{scriptsize}
\end{frame}

%%%%%%%%%%%%%%%%%%%%%%%%%%%%%%%%%%%%%%%%%%%%%%%%%%%
\begin{frame}[containsverbatim]
\frametitle{Exemplos} 
\begin{itemize}
\item Como ler uma cadeia de carateres e imprimir sua inversa? 
Assuma que a cadeia tem no m�ximo 80 caracteres.
\item {\bf N�o utilizar um vetor adicional.} 
\end{itemize}
\end{frame}

%%%%%%%%%%%%%%%%%%%%%%%%%%%%%%%%%%%%%%%%%%%%%%%%%%%
\begin{frame}[containsverbatim]
\frametitle{Exemplos} 
\begin{scriptsize}
\begin{verbatim}
#include <stdio.h>
int main(){
  char st1[80], aux;
  int i, j, tam;

  printf("Digite um texto (max. 80):");
  scanf("%s",st1);

  tam=tamanho(st1);
  
  i = 0;
  j = tam -1;
  while(i < j){
    aux = st1[i];
    st1[i] = st1[j];
    st1[j] = aux;
    i++; j--;
  }
  
  printf("A inversa:%s\n",st1);
  
  return 0;
}
\end{verbatim}
\end{scriptsize}
\end{frame}

%%%%%%%%%%%%%%%%%%%%%%%%%%%%%%%%%%%%%%%%%%%%%%%
%%%%%%%%%%%%%%%%%%%%%%%%%%%%%%%%%%%%%%%%%%%%%%%
%%%%%%%%%%%%%%%%%%%%%%%%%%%%%%%%%%%%%%%%%%%%%%%
%%%%%%%%%%%%%%%%%%%%%%%%%%%%%%%%%%%%%%%%%%%%%%%
%%%%%%%%%%%%%%%%%%%%%%%%%%%%%%%%%%%%%%%%%%%%%%%
%%%%%%%%%%%%%%%%%%%%%%%%%%%%%%%%%%%%%%%%%%%%%%%
%%%%%%%%%%%%%%%%%%%%%%%%%%%%%%%%%%%%%%%%%%%%%%%
\section{Inicializa\c{c}�o}
\begin{frame}[containsverbatim]
  \frametitle{Inicializa��o}

\begin{itemize}
\item {\it Cadeias de caracteres} podem ser definidas diretamente.
\begin{block}{Exemplo}
char frase[100] = "sim isto � poss�vel";\\
\end{block}
\end{itemize}
\end{frame}


\section{Exerc�cios}
\begin{frame}[containsverbatim]
\frametitle{Exerc�cios} 
\begin{enumerate}
\item Escreva uma fun\c{c}�o que recebe uma cadeia de caracteres contendo uma frase e 
imprime a �ltima palavra desta frase. As palavras, neste caso, s�o separadas por sinais (',', '.', ';', '!' etc) ou espa\c{c}os.

\item Escreva uma fun\c{c}�o que recebe uma cadeia de caracteres contendo uma express�o matem�tica e verifica se os par�nteses foram colocados adequadamente. Na express�o ``2*(2+(3-8)))'', por exemplo, h� um um `)'- fecha-par�ntese - sobrando.
\end{enumerate}
\end{frame}

\end{document}
