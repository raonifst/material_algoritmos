\documentclass[12pt]{article}
 
\usepackage[margin=1in]{geometry} 
\usepackage{amsmath,amsthm,amssymb}
\usepackage{hyperref} 
 
\newcommand{\N}{\mathbb{N}}
\newcommand{\Z}{\mathbb{Z}}
 
\newenvironment{theorem}[2][Theorem]{\begin{trivlist}
\item[\hskip \labelsep {\bfseries #1}\hskip \labelsep {\bfseries #2.}]}{\end{trivlist}}
\newenvironment{lemma}[2][Lemma]{\begin{trivlist}
\item[\hskip \labelsep {\bfseries #1}\hskip \labelsep {\bfseries #2.}]}{\end{trivlist}}
\newenvironment{exercise}[2][Exercise]{\begin{trivlist}
\item[\hskip \labelsep {\bfseries #1}\hskip \labelsep {\bfseries #2.}]}{\end{trivlist}}
\newenvironment{problem}[2][Problem]{\begin{trivlist}
\item[\hskip \labelsep {\bfseries #1}\hskip \labelsep {\bfseries #2.}]}{\end{trivlist}}
\newenvironment{question}[2][Question]{\begin{trivlist}
\item[\hskip \labelsep {\bfseries #1}\hskip \labelsep {\bfseries #2.}]}{\end{trivlist}}
\newenvironment{corollary}[2][Corollary]{\begin{trivlist}
\item[\hskip \labelsep {\bfseries #1}\hskip \labelsep {\bfseries #2.}]}{\end{trivlist}}
 
\begin{document}
 
% --------------------------------------------------------------
%                         Start here
% --------------------------------------------------------------
 
\title{Universidade Federal de Mato Grosso\\
Instituto de Engenharia}




\author{Algoritmos e Programa\c{c}\~ao de Computadores \\
{\bf Planejamento da Disciplina e Crit\'erios de Avalia\c{c}\~ao}} %if necessary, replace with your course title
\date{Raoni Florentino da Silva Teixeira}
\maketitle

\section{Aulas}

Curso te\'orico-pr\'atico composto por quatro aulas semanais de uma hora-aula cada. As duas aulas te\'oricas ser\~ao ministradas na {\tt SALA 11} do Bloco did\'atico I. As aulas pr\'aticas acontecer\~ao no {\tt Laborat\'orio 01} (Alan Turing) do Instituto de Computa\c{c}\~ao. 

\section{Ementa}

Conceitos b\'asicos de organiza\c{c}\~ao de computadores. Constru\c{c}\~ao de algoritmos e sua representa\c{c}\~ao em pseudoc\'odigo e linguagens de alto n\'ivel. 
Desenvolvimento sistem\'atico e implementa\c{c}\~ao de programas. 
Algoritmos Iterativos e Recursivos.
Estrutura\c{c}\~ao, depura\c{c}\~ao, testes e documenta\c{c}\~ao de programas.
Resolu\c{c}\~ao de problemas.

\section{Programa da Disciplina}

\noindent
Unidade I - Conceitos Iniciais (8 horas)
\begin{itemize}
\item Organiza\c{c}\~ao de computadores
\item Modelo computacional
\item Defini\c{c}\~ao formal de algoritmo 
\item Aspectos de modelagem de problemas
\end{itemize}

\noindent
Unidade II - Vari\'avel, express\~oes, atribui\c{c}\~ao, entrada e sa\'ida (4 horas)
\begin{itemize}
\item Vari\'aveis 
\item Express\~oes aritm\'eticas
\item Atribui\c{c}\~oes
\item Comandos de Entrada e Sa\'ida
\end{itemize}

\noindent
Unidade III - Estruturas de Sele\c{c}\~ao (4 horas)
\begin{itemize}
\item Conceitos
\item Express\~oes relacionais e l\'ogicas
\item Comandos simples (if e else)
\item Comandos Encaixados
\item Teste de algoritmos condicionais
\end{itemize}
\noindent
Unidade IV - Estrutura de repeti\c{c}\~ao (8 horas)
\begin{itemize}
\item Conceitos
\item Comandos simples (for e while)
\item Comandos encaixados
\item Teste de algoritmos iterativos
\end{itemize}

\noindent
Unidade V - Fun\c{c}\~oes (4 horas)
\begin{itemize}
\item Conceito
\item Sintaxe
\item Chamada, declara\c{c}\~ao, par\^ametros, corpo e retorno
\item Teste de algoritmos com fun\c{c}\~oes
\end{itemize}

\noindent
Unidade VI - Estruturas homog\^eneas (12 horas)
\begin{itemize}
\item Vetores
\item Busca sequencial
\item Ordena\c{c}\~ao (sele\c{c}\~ao e inser\c{c}\~ao)
\item Matrizes
\item Cadeias de caracteres
\end{itemize}

\noindent
Unidade VII – Recursividade (12 horas)
\begin{itemize}
\item Conceito
\item Fun\c{c}\~oes recursivas
\item Divis\~ao e conquista
\item Busca bin\'aria
\item Ordena\c{c}\~ao (separa\c{c}\~ao e intercala\c{c}\~ao) 
\item Teste de algoritmos recursivos
\end{itemize}

\noindent
Unidade VIII - T\'opicos adicionais (12 horas)
\begin{itemize}
\item Estruturas n\~ao homog\^eneas (struct e union)
\item No\c{c}\~oes sobre armazenamento secund\'ario
\item Apontadores
\item Leiaute e organiza\c{c}\~ao de c\'odigo
\end{itemize}

\section{Listas de Exerc\'icios}
Listas de exerc\'icios ser\~ao oferecidas como sugest\~ao para estudo {extra-classe}. Estes exer\-c\'i\-cios n\~ao ser\~ao cobrados e n\~ao entrar\~ao no c\^omputo da avalia\c{c}\~ao do aluno. Entretanto, recomenda-se fortemente aos alunos que
os fa\c{c}am como parte do estudo individual. Os exerc\'icios n\~ao ter\~ao seus gabaritos publicados. Se um aluno tiver d\'uvida em seus exerc\'icios, ou na sua resolu\c{c}\~ao, pode (e deve) utilizar os hor\'arios de atendimento para solucion\'a-la.

\section{Avalia\c{c}\~oes Pr\'aticas}
Haver\'a quatro atividades pr\'aticas ($TP_1$, $TP_2$, $TP_3$ e $TP_4$), nas quais o aluno implementar\'a programas em C que resolvem  problemas propostos.  

A m\'edia das avalia\c{c}\~oes pr\'aticas (MP) \'e dada por:

\begin{equation}
\mathrm{MP} = \frac{TP_1 + TP_2 + TP_3 + 3TP_4}{6}.
\end{equation}

\section{Avalia\c{c}\~oes Te\'oricas}
Haver\'a três provas te\'oricas ($PT_1$, $PT_2$ e $PT_3$). Estas avalia\c{c}\~oes ser\~ao realizadas sem consulta e ter\~ao dura\c{c}\~ao de cem minutos.  

A m\'edia das avalia\c{c}\~oes te\'oricas (MT) \'e dada por:

\begin{equation}
\mathrm{MT} = \frac{PT_1 + 2PT_2 + 3TP_3}{6}.
\end{equation}

Al\'em disto, ser\~ao realizados tr\^es desafios algor\'itmicos ($D1$, $D2$ e $D3$). 
Cada um destes desafios ser\'a realizado antes de uma prova te\'orica. Dependendo da solu\c{c}\~ao do desafio, at\'e no m\'aximo, um ponto ser\'a acrescentado na prova correspondente.

\section{Avalia\c{c}\~ao }

A m\'edia do semestre ($MS$) ser\'a calculada utilizando-se as m\'edias de avalia\c{c}\~oes te\'oricas e pr\'aticas.
A f\'ormula para o c\'alculo de $MS$ \'e dada por:

\begin{equation}
\mathrm{MS} = \frac{2MT + MP}{3}.
\end{equation}

%O aluno aprova-se se $MS \geq 7.0$ e sua m\'edia final ($MF$) é igual \`a $MS$. 

Uma prova final ($PF$) dever\'a ser realizada por todos alunos cuja m\'edia do semestre $MS$ \'e maior ou igual \`a tr\^es e menor que sete.
Ficam, portanto, {\bf impedidos} de fazer a $PF$ os alunos para os quais $MS  < 3.0$. A m\'edia final \'e obtida por:
\begin{equation} 
\mathrm{MF} = \begin{cases} MS, & \mbox{se } MS < 3 \\ MS, & \mbox{se } MS \geq 7 \\ \frac{MS+PF}{2}, & \mbox{caso contr\'ario.}  \end{cases}
\end{equation}

{\bf Aprovam-se os alunos cuja m\'edia final \'e maior que cinco}. 



\section{Datas Importantes}
\begin{itemize}
\item 16/09/2014 - In\'icio das Aulas.
\item 17/11/2014 - Primeira Prova Te\'orica
\item 16/12/2014 - Segunda Prova Te\'orica
\item 09/02/2014 - Terceira Prova Te\'orica
\item 19/02/2014 - Prova Final
\end{itemize}

\section{Bibliografia B\'asica}
\begin{itemize}
\item Medina, M. Fertig. \textbf{Algoritmos e Programa\c{c}\~ao: Teoria e Pr\'atica}. Novatec, 2005.
\item Ascencio, A. F. G. Campos, E. A. V. \textbf{Fundamentos da Programa\c{c}\~ao de Computadores: Algoritmos, Pascal, C/C++ e Java}. Pearson, 2007.
\item Kernighan, B. W. Ritchie, D. M. \textbf{C: a Linguagem de Programa\c{c}\~ao Padr\~ao ANSI}. Campus, 1989.
\end{itemize}

\section{Bibliografia Complementar}
\begin{itemize}
\item D. Piva Jr, A. M. Engelbrecht, G. S. Nakamiti, F. Bianchi. \textbf{Algoritmos e Programa\c{c}\~ao de Computadores}, Campus, 2012.
\item P. Feofiloff. \textbf{Algoritmos em Linguagem C}, Campus, 2009.
\item G. M. Schneider, J. Gersting. \textbf{Invitation to Computer Science}, 6. ed., 2013.
\item H. M. Deitel, P. J. Deitel. \textbf{C: Como Programar}, Prentice Hall, 2011.
\item F. Mokarze, N. Soma. \textbf{Introdu\c{c}\~ao \`a Ci\^encia da Computa\c{c}\~ao}, Campus, 2008.
\item N. Ziviani. \textbf{Projeto de Algoritmos}, Thomson, 2004.
\item Teixeira, R. F. S. \textbf{Site da Disiciplina}. Disponível em: 
 \url{http://raoniteixeira.wix.com/home\#!algoritmos/cjpl}
\end{itemize}

\end{document}