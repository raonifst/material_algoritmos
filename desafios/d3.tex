\documentclass[12pt]{article}
 
\usepackage[margin=1in]{geometry} 
\usepackage{amsmath,amsthm,amssymb}
 
\newcommand{\N}{\mathbb{N}}
\newcommand{\Z}{\mathbb{Z}}
 
\newenvironment{theorem}[2][Theorem]{\begin{trivlist}
\item[\hskip \labelsep {\bfseries #1}\hskip \labelsep {\bfseries #2.}]}{\end{trivlist}}
\newenvironment{lemma}[2][Lemma]{\begin{trivlist}
\item[\hskip \labelsep {\bfseries #1}\hskip \labelsep {\bfseries #2.}]}{\end{trivlist}}
\newenvironment{exercise}[2][Exercise]{\begin{trivlist}
\item[\hskip \labelsep {\bfseries #1}\hskip \labelsep {\bfseries #2.}]}{\end{trivlist}}
\newenvironment{problem}[2][Problem]{\begin{trivlist}
\item[\hskip \labelsep {\bfseries #1}\hskip \labelsep {\bfseries #2.}]}{\end{trivlist}}
\newenvironment{question}[2][Question]{\begin{trivlist}
\item[\hskip \labelsep {\bfseries #1}\hskip \labelsep {\bfseries #2.}]}{\end{trivlist}}
\newenvironment{corollary}[2][Corollary]{\begin{trivlist}
\item[\hskip \labelsep {\bfseries #1}\hskip \labelsep {\bfseries #2.}]}{\end{trivlist}}
 
\begin{document}
 
% --------------------------------------------------------------
%                         Start here
% --------------------------------------------------------------
 
\title{Universidade Federal de Mato Grosso\\
Instituto de Engenharia}




\author{Algoritmos e Programa\c{c}\~ao de Computadores - 2014/2 \\
{\bf Desafio 3 - Problema do Peso}} %if necessary, replace with your course title
\date{30 de novembro de 2014}
\maketitle


\section{Enunciado}
Considere o problema da pesagem enunciado a seguir. 
H\'a uma conjunto de {\bf 31 caixas} organizadas em {\bf ordem} pelo peso. Mostre como verificar se {\bf duas} destas {\bf 31 caixas} tem o mesmo peso de {\bf um determinado recipiente}, utilizando uma balan\c{c}a de pratos. Observe que as caixas est\~ao ordenadas, por\'em, n\~ao h\'a indica\c{c}\~ao sobre o peso de cada uma delas. Escreva uma solu\c{c}\~ao realizando o menor n\'umero de pesagens poss\'ivel.


\section{C\'alculo da Nota}
Para encorajar a otimiza\c{c}\~ao do uso da balan\c{c}a, a nota deste desafio, $D_3$, ser\'a calculada da seguinte maneira:
\begin{equation}
D_3 = \begin{cases} 1, & \mbox{se } n \leq 29 \\ e^{\frac{-(n-30)^2}{100}}, & \mbox{caso contr\'ario,}  \end{cases}%.
\end{equation}
em que $n$ corresponde ao n\'umero de vezes em que a balan\c{c}a \'e utilizada no pior caso. Evidentemente, $n$ \'e um n\'umero natural maior que $0$. Lembre-se de que $D_3$ ser\'a acrescentada \`a nota da terceira Prova Te\'orica.


\section{Prazo de Entrega}

O texto manuscrito contendo a solu\c{c}\~ao desta atividade deve ser entregue ao professor, impreteri\-vel\-mente, at\'e o dia {\bf 1 de fevereiro de 2015.}

\end{document}